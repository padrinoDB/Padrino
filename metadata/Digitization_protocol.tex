\documentclass[]{article}
\usepackage{lmodern}
\usepackage{amssymb,amsmath}
\usepackage{ifxetex,ifluatex}
\usepackage{fixltx2e} % provides \textsubscript
\ifnum 0\ifxetex 1\fi\ifluatex 1\fi=0 % if pdftex
  \usepackage[T1]{fontenc}
  \usepackage[utf8]{inputenc}
\else % if luatex or xelatex
  \ifxetex
    \usepackage{mathspec}
  \else
    \usepackage{fontspec}
  \fi
  \defaultfontfeatures{Ligatures=TeX,Scale=MatchLowercase}
\fi
% use upquote if available, for straight quotes in verbatim environments
\IfFileExists{upquote.sty}{\usepackage{upquote}}{}
% use microtype if available
\IfFileExists{microtype.sty}{%
\usepackage{microtype}
\UseMicrotypeSet[protrusion]{basicmath} % disable protrusion for tt fonts
}{}
\usepackage[margin=1in]{geometry}
\usepackage{hyperref}
\hypersetup{unicode=true,
            pdftitle={Digitization Protocol},
            pdfauthor={Sam Levin},
            pdfborder={0 0 0},
            breaklinks=true}
\urlstyle{same}  % don't use monospace font for urls
\usepackage{graphicx,grffile}
\makeatletter
\def\maxwidth{\ifdim\Gin@nat@width>\linewidth\linewidth\else\Gin@nat@width\fi}
\def\maxheight{\ifdim\Gin@nat@height>\textheight\textheight\else\Gin@nat@height\fi}
\makeatother
% Scale images if necessary, so that they will not overflow the page
% margins by default, and it is still possible to overwrite the defaults
% using explicit options in \includegraphics[width, height, ...]{}
\setkeys{Gin}{width=\maxwidth,height=\maxheight,keepaspectratio}
\IfFileExists{parskip.sty}{%
\usepackage{parskip}
}{% else
\setlength{\parindent}{0pt}
\setlength{\parskip}{6pt plus 2pt minus 1pt}
}
\setlength{\emergencystretch}{3em}  % prevent overfull lines
\providecommand{\tightlist}{%
  \setlength{\itemsep}{0pt}\setlength{\parskip}{0pt}}
\setcounter{secnumdepth}{0}
% Redefines (sub)paragraphs to behave more like sections
\ifx\paragraph\undefined\else
\let\oldparagraph\paragraph
\renewcommand{\paragraph}[1]{\oldparagraph{#1}\mbox{}}
\fi
\ifx\subparagraph\undefined\else
\let\oldsubparagraph\subparagraph
\renewcommand{\subparagraph}[1]{\oldsubparagraph{#1}\mbox{}}
\fi

%%% Use protect on footnotes to avoid problems with footnotes in titles
\let\rmarkdownfootnote\footnote%
\def\footnote{\protect\rmarkdownfootnote}

%%% Change title format to be more compact
\usepackage{titling}

% Create subtitle command for use in maketitle
\newcommand{\subtitle}[1]{
  \posttitle{
    \begin{center}\large#1\end{center}
    }
}

\setlength{\droptitle}{-2em}
  \title{Digitization Protocol}
  \pretitle{\vspace{\droptitle}\centering\huge}
  \posttitle{\par}
  \author{Sam Levin}
  \preauthor{\centering\large\emph}
  \postauthor{\par}
  \predate{\centering\large\emph}
  \postdate{\par}
  \date{2018-03-12}


\begin{document}
\maketitle

{
\setcounter{tocdepth}{5}
\tableofcontents
}
\newpage

\includegraphics{logo-padrino-4c.png}
\includegraphics{logo-madrina-4c.png}

\newpage

\section{Padrino digitization
protocol}\label{padrino-digitization-protocol}

Welcome to the \texttt{Padrino} and \texttt{Madrina} digitization team!
The following is an in-depth guide for how to translate a published IPM
into the \emph{Excel} sheets and what to do with that sheet once you
have completed the process.

There is a good chance I will have forgotten to include some important
aspects of what this work entails. If that happens, please create an
\emph{Issue} in the \href{https://github.com/levisc8/Padrino}{GitHub
repository} as opposed to emailing me. This helps me (and, hopefully,
you) keep all discussion related to the problem in a centralized
location. Most importantly, this centralized location is accesible and
searchable by your fellow digitizers, so they can also participate in
the discussion or return to it later to reference it as needed.

This document assumes you have a general understanding of an IPM and why
we want to have all of them in one place.

\section{Data base structure}\label{data-base-structure}

\texttt{Padrino} and \texttt{Madrina} are remotely hosted relational
data bases that consider an individual IPM as the ``atomic unit'', as
opposed to each vital rate or parameter. This helps keep relations tidy
and avoid unnecessary data duplication, but can introduce some confusion
at first. Hopefully, after reading this, the reasons for this decision
will become more clear.

Relational data bases can be a bit tricky to work with if you aren't
already familiar with them. My (admittedly limited) experience has been
that this is not the most accesible format for most ecologists who are
usually more accustomed to working in \emph{R} or \emph{Excel}.
Therefore, we have created a ``flat'' version of it that is a set of 5
\emph{Excel} sheets. The columns of each one are described below. The
format of each description is: \textbf{data type}, description of
variable, \emph{constraint}.

\subsection{Metadata}\label{metadata}

This is the table you will find when opening the spreadsheet. It
contains important information about the study site, species, and
authors that aren't necessarily relevant to construction of the IPM, but
provide important context nonetheless.

\paragraph{ipm\_id}\label{ipm_id}

\textbf{Character}

This column contains a unique identifier for each IPM in the data base.
There should only be \textbf{1} row per IPM in this table.

\emph{This column cannot be blank.}

\paragraph{species\_author}\label{species_author}

\textbf{Character}

The genus and species epithet of the organism used by the author in the
publication.

\emph{This column cannot be blank.}

\paragraph{species\_accepted}\label{species_accepted}

\textbf{Character}

The currently accepted genus and species epithet of the organism. For
\texttt{Padrino} entries, this will come from the
\href{http://www.theplantlist.org/}{The Plant List}. For
\texttt{Madrina} entries, this will come from the
\href{http://www.catalogueoflife.org/}{Catalogue of Life}.

\emph{This column cannot be blank.}

\paragraph{tax\_genus}\label{tax_genus}

\textbf{Character}

The accepted genus name of the organism. For \texttt{Padrino} entries,
this will come from the \href{http://www.theplantlist.org/}{The Plant
List}. For \texttt{Madrina} entries, this will come from the
\href{http://www.catalogueoflife.org/}{Catalogue of Life}.

\emph{This column cannot be blank.}

\paragraph{tax\_family}\label{tax_family}

\textbf{Character}

The accepted family of the organism. For \texttt{Padrino} entries, this
will come from the \href{http://www.theplantlist.org/}{The Plant List}.
For \texttt{Madrina} entries, this will come from the
\href{http://www.catalogueoflife.org/}{Catalogue of Life}.

\emph{This column may be blank.}

\paragraph{tax\_order}\label{tax_order}

\textbf{Character}

The accepted order of the organism. For \texttt{Padrino} entries, this
will come from the \href{http://www.theplantlist.org/}{The Plant List}.
For \texttt{Madrina} entries, this will come from the
\href{http://www.catalogueoflife.org/}{Catalogue of Life}.

\emph{This column may be blank.}

\paragraph{tax\_class}\label{tax_class}

\textbf{Character}

The accepted class of the organism. For \texttt{Padrino} entries, this
will come from the \href{http://www.theplantlist.org/}{The Plant List}.
For \texttt{Madrina} entries, this will come from the
\href{http://www.catalogueoflife.org/}{Catalogue of Life}.

\emph{This column may be blank.}

\paragraph{tax\_phylum}\label{tax_phylum}

\textbf{Character}

The accepted phylum of the organism. For \texttt{Padrino} entries, this
will come from the \href{http://www.theplantlist.org/}{The Plant List}.
For \texttt{Madrina} entries, this will come from the
\href{http://www.catalogueoflife.org/}{Catalogue of Life}.

\emph{This column may be blank.}

\paragraph{kingdom}\label{kingdom}

\textbf{Character}

The accepted kingdom of the organism. For \texttt{Padrino} entries, this
will come from the \href{http://www.theplantlist.org/}{The Plant List}.
For \texttt{Madrina} entries, this will come from the
\href{http://www.catalogueoflife.org/}{Catalogue of Life}.

\emph{This column may be blank.}

\paragraph{organism\_type}\label{organism_type}

\textbf{Character}

\subparagraph{\texorpdfstring{\texttt{Padrino}
entries}{Padrino entries}}\label{padrino-entries}

This is the general plant/algae type. This will usually come from the
publication itself, but sometimes you may need to use other sources
(e.g.~other publications or taxonomic data bases) to find this
information. Possible values are as follows

\begin{itemize}
\item
  Algae: brown, green or red. Green algae are in the \emph{Plantae}
  kingdom, but are still considered algae for the purposes of this
  variable.
\item
  Fungi: This includes fungus species, yeasts, molds, and multicellular
  fungi.
\item
  Annual: This includes annuals and biennials. Annuals complete their
  entire lifecycle (birth, growth, reproduction, death) within a year
  wherease biennials can stretch that window to two years. For the sake
  of simplicity (and because both will die following reproduction), they
  are both classified as ``Annual'' in \texttt{Padrino}.
\item
  Bryophyte: All bryophytes.
\item
  Epiphyte: All epiphytes.
\item
  Fern: All ferns species.
\item
  Herbaceous perennial: All plants that herbaceous and have the
  potential to live for more than two years.
\item
  Liana: All lianas.
\item
  Palm: All palm species.
\item
  Shrub: Woody upright plants that are not trees or palms.
\item
  Succulent: All succulent species.
\item
  Tree: All tree species.
\end{itemize}

\subparagraph{\texorpdfstring{\texttt{Madrina}
entries}{Madrina entries}}\label{madrina-entries}

This is generally the same as \texttt{Class} for animals (except humans,
which are recorded using their genus and species epithet). Non-animal
species that are also not plants are typically recorded as
\emph{Bacteria} or \emph{Virus}.

\emph{This column may be blank}.

\paragraph{dicot\_monocot}\label{dicot_monocot}

\textbf{Character}

Indicates whether a species is a dicot or monocot. Not applicable for
\texttt{Madrina} entries.

\emph{This column may be blank.}

\paragraph{angio\_gymno}\label{angio_gymno}

\textbf{Character}

Indicates whether a species is a angiosperm or gymnosperm. Not
applicable for \texttt{Madrina} entries.

\emph{This column may be blank.}

\paragraph{authors}\label{authors}

\textbf{Character}

The last name of all authors. Multiple entries should be separated with
semicolon (``;'').

\emph{This column cannot be blank}.

\paragraph{journal}\label{journal}

\textbf{Character}

The document that the information comes from. Possible values are listed
below.

\begin{itemize}
\item
  Abbreviated name of the journal: We use the
  \href{http://cms.library.illinois.edu/export/biotech/j-abbrev.html}{BIOSIS}
  system for abbreviating journal names. More information on how to use
  it is in the link.
\item
  Book: Models are sourced from a book.
\item
  PhD Thesis: Models are sourced from a PhD thesis.
\item
  MSc Thesis: Models are sourced from an MSc thesis.
\item
  Report: Models are sourced from a report.
\item
  Conference talk: Models are sourced from a conference talk.
\item
  Conference poster: Models are sourced from a conference poster.
\end{itemize}

\paragraph{pub\_year}\label{pub_year}

\textbf{Integer}

The year that the model was published.

\emph{This column cannot be blank.}

\paragraph{doi}\label{doi}

\textbf{Character}

The \emph{DOI} or \emph{ISBN} for the publication.

\emph{This column may be blank.}

\paragraph{corresponding\_author}\label{corresponding_author}

\textbf{Character}

The author to whom correspondance should be directed.

\emph{This column may be blank.}

\paragraph{email\_year}\label{email_year}

\textbf{Character}

The email address of the corresponding author with the year it is from
in parentheses. If the email address is no longer in use, add the word
``Dead'' after a comma in the parentheses.

Example: \href{mailto:levisc8@gmail.com}{\nolinkurl{levisc8@gmail.com}}
(2018); \href{mailto:levisc8@wfu.edu}{\nolinkurl{levisc8@wfu.edu}}
(2010, Dead)

\emph{This column may be blank.}

\paragraph{remark}\label{remark}

\textbf{Character}

Any observations you have about the model that are not captured by the
other columns in \texttt{Metadata}.

\emph{This column may be blank.}

\paragraph{apa\_citation}\label{apa_citation}

\textbf{Character}

The full \href{http://www.bibme.org/apa}{APA citation} for the source.

\emph{This column may be blank.}

\paragraph{demog\_appendix\_link}\label{demog_appendix_link}

\textbf{Character}

If the model parameters are contained in an appendix, then include the
link to said appendix here.

\emph{This column may be blank.}

\paragraph{duration}\label{duration}

\textbf{Integer}

Model duration is defined as the
\texttt{end\_year\ -\ start\_year\ +\ 1}. Thus, this overlooks any years
that were skipped.

\emph{This column cannot be blank.}

\paragraph{start\_year}\label{start_year}

\textbf{Integer}

The first year of data collection for the model.

\emph{This column cannot be blank.}

\paragraph{start\_month}\label{start_month}

\textbf{Integer}

The first month of data collection for the model. Months are numbered
starting at 1 for January and continuing to 12 for December.

\emph{This column may be blank.}

\paragraph{end\_year}\label{end_year}

\textbf{Integer}

The last year of data collection for the model.

\emph{This column cannot be blank.}

\paragraph{end\_month}\label{end_month}

\textbf{Integer}

The last month of data collection for the model.

\emph{This column may be blank.}

\paragraph{periodicity}\label{periodicity}

\textbf{Decimal}

The number of times the model iterates per year. Periodic models with
two transitions per year will have a value of two. Some IPMs for long
lived tree species may have 5 year transitions; these will have a value
of 0.2.

\emph{This column may not be blank.}

\paragraph{number\_populations}\label{number_populations}

\textbf{Integer}

The number of populations described by the model.

\emph{This column may not be blank.}

\paragraph{lat}\label{lat}

\textbf{Decimal}

The decimal latitude coordinates for the site where data used in the
model was collected from. Positive values refer to the northern
hemisphere, while negative coordinates refer to the southern hemisphere.

\emph{This column may be blank.}

\paragraph{lon}\label{lon}

\textbf{Decimal}

The decimal longitude coordinates for the site where data used in the
model was collected from. Positive values refer to the eastern
hemisphere, while negative coordinates refer to the western hemisphere.

\emph{This column may be blank.}

\paragraph{altitude}\label{altitude}

\textbf{Decimal}

The altitude in meters above sea level of the site where data used in
the model was collected from.

\emph{This column may be blank.}

\paragraph{country}\label{country}

\textbf{Character}

The
\href{https://unstats.un.org/unsd/tradekb/knowledgebase/country-code}{ISO
3} country code.

\emph{This column may be blank.}

\paragraph{continent}\label{continent}

\textbf{Character}

The continent the study was conducted on. Possible values are below.

\begin{itemize}
\item
  n\_america: Includes Canada, USA, and Mexico.
\item
  s\_america: Includes everything in the Americas except for Canada,
  USA, and Mexico.
\item
  africa
\item
  asia
\item
  europe
\item
  oceania: Various definitions exist, we are using
  \href{https://en.wikipedia.org/wiki/List_of_Oceanian_countries_by_population}{this
  one}.
\item
  antarctica
\end{itemize}

\emph{This column may be blank.}

\paragraph{ecoregion}\label{ecoregion}

\textbf{Character}

Indication of the ecoregion for the study, using the categories
described in Figure 1 of Olson et al. (2001). For a more inclusive
description of water ecoregions, see
\url{http://worldwildlife.org/biomes} The one exception for this is the
code LAB used for studies conducted in laboratory or greenhouse
conditions. Possible values are below.

\begin{itemize}
\item
  TMB Terrestrial tropical and subtropical moist broadleaf forests
\item
  TDB Terrestrial tropical and subtropical dry broadleaf forests
\item
  TSC Terrestrial tropical and subtropical coniferous forests
\item
  TBM Terrestrial temperate broadleaf and mixed forests
\item
  TCF Terrestrial temperate coniferous forests
\item
  BOR Terrestrial boreal forests/ taiga
\item
  TGV Terrestrial tropical and subtropical grasslands, savannas and
  shrublands
\item
  TGS Terrestrial temperate grasslands, savannas, and shrublands
\item
  FGS Terrestrial flooded grasslands and savannas
\item
  MON Terrestrial montane grasslands and shrublands
\item
  TUN Terrestrial -- tundra
\item
  MED Terrestrial Mediterranean forests, woodlands and scrubs
\item
  DES Terrestrial deserts and xeric shrublands
\item
  MAN Terrestrial -- mangroves
\item
  LRE Freshwater large river ecosystems
\item
  LRH Freshwater large river headwater ecosystems
\item
  LRD Freshwater large river delta ecosystems
\item
  SRE Freshwater small river ecosystems
\item
  SLE Freshwater small lake ecosystems
\item
  LLE Freshwater large lake ecosystems
\item
  XBE Freshwater xeric basin ecosystems
\item
  POE Marine polar ecosystems
\item
  TSS Marine temperate shelf and seas ecosystems
\item
  TEU Marine temperate upwellings
\item
  TRU Marine tropical upwellings
\item
  TRC Marine tropical coral
\item
  LAB Laboratory or greenhouse conditions -- controlled, usually indoor,
  conditions that mean the study species is not affected by the
  environment conditions typical of the actual geographic location of
  the study
\end{itemize}

\emph{This column may be blank.}

\paragraph{studied\_sex}\label{studied_sex}

\textbf{Character}

The sex of the individuals modeled.

\begin{itemize}
\item
  M Studied only males
\item
  F Studied only females
\item
  H Studied hermaphrodites
\item
  M/F Males and females separately in the same IPM
\item
  A All sexes modeled together
\end{itemize}

\emph{This column may be blank.}

\paragraph{eviction\_used}\label{eviction_used}

\textbf{Boolean}

Indicates whether authors corrected their discretized kernels for
eviction
(\href{https://scholarship.rice.edu/bitstream/handle/1911/69874/Avoiding\%20unintentional\%20eviction.pdf?sequence=1}{Williams
et al. 2012}). Possible values are \emph{t} and \emph{f}. \emph{t}
indicates eviction was corrected for, \emph{f} indicates that it was
not.

\emph{This column cannot be blank.}

\paragraph{evict\_type}\label{evict_type}

\textbf{Character}

The type of correction used for eviction. Still working out the
convention to use for this\ldots{}.

\emph{This column may be blank.}

\paragraph{treatment}\label{treatment}

\textbf{Character}

If a treatment was applied to the modeled population, indicate that
here.

\emph{This column may be blank}.

\subsection{States}\label{states}

This table contains information on the state variables used by the
authors to generate their IPM.

\paragraph{ipm\_id}\label{ipm_id-1}

\textbf{Character}

This column contains a unique identifier for each IPM in the data base.
However, this differs from the \texttt{Metadata} table in that a single
IPM may have multiple rows in this table.

\emph{This column cannot be blank.}

\paragraph{state\_variable}\label{state_variable}

\textbf{Character}

This column contains the name of a state variable used to construct an
IPM \emph{as reported by the authors of the paper}. For example, this
could be \textbf{DBH} for a tree species or \textbf{Body\_Mass} for an
animal species. State variables do not need to be continuous. Examples
of discrete state variables include reproductive status, age, or
pathogen load (e.g.~low, medium, high). A single model may use multiple
state variables (and thus have multiple rows in this table, see above).

\emph{This column cannot be blank.}

\paragraph{discrete}\label{discrete}

\textbf{Boolean}

This column indicates whether or not the variable is discrete or
continuous. Use \texttt{"t"} to indicate a variable is discrete and
\texttt{"f"} to indicate that it is continuous.

\emph{This column cannot be blank.}

\paragraph{discrete\_type}\label{discrete_type}

\textbf{Character}

This column indicates the type of discretization used for a discrete
state variable. In depth explanations are provided in the
\protect\hyperlink{discrete-vars}{Discrete Variables} appendix.

\emph{This column may be blank.}

\subsection{Domains}\label{domains}

This table contains information on the domains associated with each
state variable in \texttt{States}. Keep in mind that one
\texttt{state\_variable} can have multiple domains (which themselves may
be defined by a different \texttt{state\_variable}!).

\paragraph{ipm\_id}\label{ipm_id-2}

\textbf{Character}

This column contains a unique identifier for each IPM in the data base.
However, this differs from the \texttt{Metadata} table in that a single
IPM may have multiple rows in this table.

\emph{This column cannot be blank.}

\paragraph{state\_variable}\label{state_variable-1}

\textbf{Character}

This column contains the same state variables as in the \texttt{States}
table. However, a given \texttt{state\_variable} may have multiple
domains, so entries in this column need not be unique.

\emph{This column cannot be blank.}

\paragraph{domain}\label{domain}

\textbf{Character}

This column contains a unique identifier for the domain of each
\texttt{state\_variable}. For example, if \textbf{DBH} is implemented on
3 separate domains in the publication (perhaps for 3 different light
environments in some megamatrix), then this could be named
\textbf{size1}, \textbf{size2}, and \textbf{size3} for each one. The
corresponding rows in \texttt{state\_variable} should be filled in with
\textbf{DBH} (i.e.~not unique).

\paragraph{lower}\label{lower}

\textbf{decimal}

In most cases, this will be a decimal or integer corresponding to the
smallest value of the \texttt{state\_variable} in the given domain.

\paragraph{upper}\label{upper}

\paragraph{n\_meshpoints}\label{n_meshpoints}

\textbf{Integer}

The number of bins that each domain is divided into for numerical
integration. For some discrete variables, this will be blank.

\emph{This column may be blank.}

\subsection{\texorpdfstring{Model Expressions
(\texttt{ModelExpr})}{Model Expressions (ModelExpr)}}\label{model-expressions-modelexpr}

This table contains textual expressions of the models used to create
IPM. A given IPM will have many rows in this table.

\paragraph{ipm\_id}\label{ipm_id-3}

\textbf{Character}

This column contains a unique identifier for each IPM in the data base.
However, this differs from the \texttt{Metadata} table in that a single
IPM will have multiple rows in this table.

\emph{This column cannot be blank.}

\paragraph{demographic\_parameter}\label{demographic_parameter}

\paragraph{formula}\label{formula}

See \protect\hyperlink{model-forms}{Writing Model Formulae} for
additional details

\paragraph{model\_type}\label{model_type}

\paragraph{model\_family}\label{model_family}

\paragraph{kernel\_id}\label{kernel_id}

\subsection{\texorpdfstring{Model Values
(\texttt{ModelValues})}{Model Values (ModelValues)}}\label{model-values-modelvalues}

This contains the actual values for all of the parameters described in
\texttt{ModelExpr}. A single IPM will have many rows in this table.

Every column in this table \textbf{must} be filled in to be able to
enter the data base.

\paragraph{ipm\_id}\label{ipm_id-4}

\textbf{Character}

This column contains a unique identifier for each IPM in the data base.
However, this differs from the \texttt{Metadata} table in that a single
IPM will have multiple rows in this table.

\emph{This column cannot be blank.}

\paragraph{demographic\_parameter}\label{demographic_parameter-1}

\paragraph{state\_variable}\label{state_variable-2}

\paragraph{parameter\_type}\label{parameter_type}

\paragraph{parameter\_name}\label{parameter_name}

\paragraph{parameter\_value}\label{parameter_value}

\hypertarget{model-forms}{\section{Writing Model
Formulae}\label{model-forms}}

Details on conventions and lots of examples!

\hypertarget{discrete-vars}{\section{Discrete Variables in Padrino and
Madrina}\label{discrete-vars}}

The flexibility of IPMs to allow classification of individuals as a
function of both continuous and discrete characters is one reason they
are so powerful. Unfortunately, it also makes describing them in a data
base a bit more complicated. Currently, there are a couple of supported
types of discrete variables.

An IPM can be comprised of some number smaller subcomponents. For this
data base, we describe each one separately in \texttt{ModelExpr} and
then supply the \texttt{discrete\_type} tag in the \texttt{States} table
to indicate how those subcomponents need to be combined later on. I will
attempt to describe each \texttt{discrete\_type} in depth below.

\begin{itemize}
\item
  \textbf{Lefkovitch:} This \texttt{discrete\_type} encompasses
  instances where authors have constructed a block megamatrix with
  components that are themselves IPMs. For example, Metcalf et al.
  (2009) generate a 6 x 6 megamatrix with each element corresponding to
  an IPM that describes transition probabilities from a given size and
  canopy illumination environment at time \emph{t} to a givent size and
  canopy illumination environment at time \emph{t+1}. Blocks on the
  diagonal of this matrix represent survival and growth transitions for
  trees that remain in the same light environment through time.
  Sub-diagonal elements represent survival and growth transitions for
  trees that move into more shaded environments, while super-diagonal
  represent survival and growth transitions for trees that move into
  lighter environments.

  In general, \texttt{discrete\_types} of \textbf{Lefkovitch} will be
  used to represent variables where any transition between discrete
  states is possible.
\item
  \textbf{Leslie/Age:} This \texttt{discrete\_type} describes IPMs where
  the discrete variable of interest is ordered an individual
  \textbf{must} transition to the next the next value of the discrete
  variable if it survives through the projection interval. The most
  common example of this would be an age x size IPM (e.g.~Childs et al.
  2003). This can be represented as a block megamatrix where the top row
  of blocks represents age and size dependent fecundity and sub-diagonal
  blocks represent age and size depedent growth and survival. All other
  blocks are filled with 0s. This differs from the \textbf{Lefkovitch}
  designation because not all block-level transitions are possible
  (i.e.~an individual cannot become 5 years old in at \emph{t+1} if it
  is only 2 years old at \emph{t}).
\item
\end{itemize}


\end{document}
