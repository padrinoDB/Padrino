\documentclass[]{article}
\usepackage{lmodern}
\usepackage{amssymb,amsmath}
\usepackage{ifxetex,ifluatex}
\usepackage{fixltx2e} % provides \textsubscript
\ifnum 0\ifxetex 1\fi\ifluatex 1\fi=0 % if pdftex
  \usepackage[T1]{fontenc}
  \usepackage[utf8]{inputenc}
\else % if luatex or xelatex
  \ifxetex
    \usepackage{mathspec}
  \else
    \usepackage{fontspec}
  \fi
  \defaultfontfeatures{Ligatures=TeX,Scale=MatchLowercase}
\fi
% use upquote if available, for straight quotes in verbatim environments
\IfFileExists{upquote.sty}{\usepackage{upquote}}{}
% use microtype if available
\IfFileExists{microtype.sty}{%
\usepackage{microtype}
\UseMicrotypeSet[protrusion]{basicmath} % disable protrusion for tt fonts
}{}
\usepackage[margin=1in]{geometry}
\usepackage{hyperref}
\hypersetup{unicode=true,
            pdftitle={Writing IPM Formulae},
            pdfauthor={Sam Levin},
            pdfborder={0 0 0},
            breaklinks=true}
\urlstyle{same}  % don't use monospace font for urls
\usepackage{graphicx,grffile}
\makeatletter
\def\maxwidth{\ifdim\Gin@nat@width>\linewidth\linewidth\else\Gin@nat@width\fi}
\def\maxheight{\ifdim\Gin@nat@height>\textheight\textheight\else\Gin@nat@height\fi}
\makeatother
% Scale images if necessary, so that they will not overflow the page
% margins by default, and it is still possible to overwrite the defaults
% using explicit options in \includegraphics[width, height, ...]{}
\setkeys{Gin}{width=\maxwidth,height=\maxheight,keepaspectratio}
\IfFileExists{parskip.sty}{%
\usepackage{parskip}
}{% else
\setlength{\parindent}{0pt}
\setlength{\parskip}{6pt plus 2pt minus 1pt}
}
\setlength{\emergencystretch}{3em}  % prevent overfull lines
\providecommand{\tightlist}{%
  \setlength{\itemsep}{0pt}\setlength{\parskip}{0pt}}
\setcounter{secnumdepth}{0}
% Redefines (sub)paragraphs to behave more like sections
\ifx\paragraph\undefined\else
\let\oldparagraph\paragraph
\renewcommand{\paragraph}[1]{\oldparagraph{#1}\mbox{}}
\fi
\ifx\subparagraph\undefined\else
\let\oldsubparagraph\subparagraph
\renewcommand{\subparagraph}[1]{\oldsubparagraph{#1}\mbox{}}
\fi

%%% Use protect on footnotes to avoid problems with footnotes in titles
\let\rmarkdownfootnote\footnote%
\def\footnote{\protect\rmarkdownfootnote}

%%% Change title format to be more compact
\usepackage{titling}

% Create subtitle command for use in maketitle
\newcommand{\subtitle}[1]{
  \posttitle{
    \begin{center}\large#1\end{center}
    }
}

\setlength{\droptitle}{-2em}

  \title{Writing IPM Formulae}
    \pretitle{\vspace{\droptitle}\centering\huge}
  \posttitle{\par}
    \author{Sam Levin}
    \preauthor{\centering\large\emph}
  \postauthor{\par}
      \predate{\centering\large\emph}
  \postdate{\par}
    \date{January 12, 2019}


\begin{document}
\maketitle

{
\setcounter{tocdepth}{3}
\tableofcontents
}
\section{General Overview}\label{general-overview}

\texttt{Padrino} uses its own type of notation that is somewhat R-like,
but still general enough that users of other languages could adapt it on
their own. It uses text strings of the formula for each successive level
of parameters in an IPM (constants, vital rates, kernels) to represent
the underlying models and generate the correct iteration matrices. The
notation requires some practice to get right and has a few quirks in it
due to the nature of working with strings in R/Perl.

Briefly, this document uses the following format to distinguish between
tables in the database, columns in a table, and entries in a column:

\textbf{Table in the database}

\emph{Column in a table}

\texttt{Entry\ in\ a\ column}

This document generally only applies to the following tables in the
database: \textbf{IpmKernels}, \textbf{VitalRateExpr},
\textbf{ParameterValues}, and \textbf{EnvironmentaVariables}.
Occasionally, other tables will be referenced though. They will be
highlighted using the same font scheme as above.

\section{The Formulae}\label{the-formulae}

In general, The \emph{formula} column of \textbf{IpmKernels} and
\textbf{VitalRateExpr} will always have a Left Hand Side (abbreviated w/
LHS) and a Right Hand Side (RHS). Getting these correct is critical to
the functioning of the database. The LHS and RHS have slightly different
notation schemes at each level.

Explanations for how to enter each type of formula/expression are given
in this section. Additionally, there are worked examples in the next
section which will hopefully help to clarify any confusion in the
definitions. If you do find yourself confused, please email me at
\href{mailto:levisc8@gmail.com}{\nolinkurl{levisc8@gmail.com}} or file
an issue in the project's github repository
(\url{https://github.com/levisc8/Padrino/issues}) so that I can work on
improving this document! I'll also try to provide a more personalized
introduction to this rather idiosyncratic topic.

\subsection{Constants}\label{constants}

\subsubsection{In ParamateverValues}\label{in-paramatevervalues}

All entries in \textbf{ParameterValues} should be constants (e.g.~there
is no additional function stored away somewhere that calculates them).
Thus, this is the exciting part where you get to choose names for all of
the parameters in your model! Yippee!! Thus, the column
\texttt{parameter\_name} should just contain the name of the parameter;
no LHS or RHS. All terms in this table should appear in the RHS of an
expression in one of the other tables.

\subsubsection{In VitalRateExpr and
IpmKernels}\label{in-vitalrateexpr-and-ipmkernels}

All constants that \textbf{do not} appear in a \textbf{VitalRateExpr}
and only appear in an \textbf{IpmKernel} need to be written as follows
in \textbf{VitalRateExpr} \emph{formula}:

\texttt{parameter\_name\ =\ parameter\_name} where parameter\_name is
the name of the parameter taken from the \textbf{ParameterValues} table.

This ensures that all kernels that use that value can access it when it
comes time to build them.

\subsubsection{In EnvironmetVariables}\label{in-environmetvariables}

\textbf{EnvironmentVariables} is a catch-all table that describes the
following types of IPMs: ones with hierarchical models (e.g.~with random
effects/discrete fixed effects beyond just the state variable),
environmentally dependent IPMs ( e.g.~contains terms for temperature or
precipitation in their vital rate expressions), and density dependent
IPMs (where vital rates are at least partially a function of the
number/state structure of individuals at each time step). Currently,
only the first type are implemented, so this notation may change.
Beware!

For hierarchical models, this table stores the different levels of the
hierarchical effect and provides information on suffixes that these
values are meant to replace. Temporal effects (e.g.~years of sampling)
and spatial effects (e.g.~plot names) are the most common types of
hierarchical effects, but others will appear as well.

\emph{env\_variable} provides a description of the effect. Using the two
examples above, one could enter \texttt{year} or \texttt{plot} in this
column.

\emph{vr\_expr\_name} gives the suffix that is used to denote this
effect in other parts of the database. This is appended to the name of
the continuous state variable , so if the state variable in the model
was \texttt{size}, then this could be entered as \texttt{size\_yr} or
\texttt{size\_pl}. If a model contains multiple hierarchical effects,
then each one gets its own row in the table.

Unforunately, \emph{env\_range} introduces a bit of complexity.
Hierarchical effects are generally discrete, may or may not have a set
order, and ordered effects may or may not be easily represented by a
sequence of integers. Below we cover both cases.

For effects that have an ordered sequence to them (e.g.~years of
sampling), there are two different ways to enter them, depending on
whether the ordering is consecutive or non-consecutive. If the order is
consecutive and is easily represented with integers (e.g.~years of
sampling), they can be entered using the following notation:
\texttt{year\_1:year\_n-1)} where year\_1 is the first year of sampling
and year\_n-1 is the second to last year of sampling.

For effects that are not ordered or are not easily represented with a
sequence of consecutive integers, then the following notation is used:
\texttt{c("level\_A",\ "level\_B",\ "level\_C",\ "Level\_...")}. Note
that each level is now encapsulated with quotation marks, all levels are
separated by commas, and everything is wrapped in \texttt{c()}. This is
very important (though perhaps a bit lazy of me and the least
``language-agnostic'' aspect of the database).

The above is a bit vague and confusing, so see the
\protect\hyperlink{examples}{Worked Examples} in the next section for a
more concrete demonstration of how this works.

\subsubsection{Why all this confusion for hierarchical
models?}\label{why-all-this-confusion-for-hierarchical-models}

The general goal of the framework we implemented for hierarchical models
is to represent them as closely to how they are defined as possible.
This has the additional benefit of greatly reducing the amount of typing
required to enter them, consequently lowering the risk of typos. A
separate PDF explains how this substitution works and can be found
\href{https://github.com/levisc8/Padrino/blob/master/metadata/Hierarchical_effects_subbing_schematic.pdf}{here}.

Of course, there's nothing quite like diving into the source code to
understand the logic of it all! The code that deals with the
split-duplicate-combining of hierarchical models can be found
\href{https://github.com/levisc8/RPadrino/blob/master/R/split_ranefs.R}{here}
(top level function) and
\href{https://github.com/levisc8/RPadrino/blob/master/R/split_ranefs_helpers.R}{here}
(internal helpers).

\subsection{Vital rates}\label{vital-rates}

\subsubsection{In VitalRateExpr}\label{in-vitalrateexpr}

The following is relevant to \emph{formula} and \emph{kernel\_id}.

\emph{formula} should contain a textual representation of the vital rate
model. The LHS of the model should include the name of the vital rate
and have the state variable that is a function of in parentheses (e.g.
\texttt{surv(size\_1)\ =\ ...}). The RHS of the model should contain the
coefficients and mathematical transformations needed to produce a single
value on the LHS in the form of a mathematical expression (e.g.
\texttt{1\ /\ (1\ +\ exp(-(surv\_int\ +\ surv\_slope\ *\ size\_1)))}.
The inverse logit transformation is performed on the RHS of the
expression as opposed to the LHS (as it often appears in papers). Thus,
the complete entry in \emph{formula} becomes

\texttt{surv(size\_1)\ =\ 1\ /\ (1\ +\ exp(-(surv\_int\ +\ surv\_slope\ *\ size\_1)))}

Some vital rates may contain probability density functions (PDFs). Each
density function has a specific abbreviation that we use and these
expression wrap their arguments in parentheses. For example, a growth
function may be entered as:

\texttt{g(size\_2,\ size\_1)\ =\ Norm(mean\_growth,\ sd\_growth)}

Vital rates that are modeled with hierarchical effects get the same
suffixed notation that was described above. For example, if the survival
model from above was to include a random intercept for year (abbreviated
\texttt{yr} in **EnvironmentVariables), then it would be re-written as

\texttt{s\_yr(size\_1)\ =\ 1\ /\ (1\ +\ exp(-(surv\_int\_yr\ +\ surv\_slope\ *\ size\_1)))}

Notice that the only change we made was to insert the \texttt{\_yr}
suffix to the parameters that were modified by the hierarchical effect.

See the \protect\hyperlink{transforms-funs}{Common Transformations and
Functions} section for some examples of common mathematical
transformations that appear in here. Additionally, see the
\protect\hyperlink{pdf-abbrev}{PDF Abbreviations} sections for a
complete list of probability density functions that we support and their
associated abbreviations.

For models without any sort of hierarchical effects, \emph{kernel\_id}
is just the name of the kernel (e.g. \texttt{K\_all;\ P;\ F}).

For models with hierarchical effects, the suffix from the
\textbf{EnvironmentVariables} \emph{vr\_expr\_name} is appended to the
typical kernel id. For example, if the random year effect is abbreviated
with \texttt{yr}, then the \emph{kernel\_id}s become
\texttt{K\_all\_yr;\ P\_yr;\ F\_yr}, etc. For models with multiple
hierarchical effects, for example multiple plots and many years, then
both suffixes are appended to each \emph{kernel\_id} so the above
becomes \texttt{K\_all\_yr\_pl;\ P\_yr\_pl;\ F\_yr\_pl}.

Note that regardless of whether there are hierarchical effects or not,
vital rates that appear in multiple kernels are only written once and
are always separated by a semicolon (the ``;'').

\subsubsection{In IpmKernels}\label{in-ipmkernels}

Vital rates are denoted in \textbf{IpmKernels} using the LHS of the
\emph{formula} from \textbf{VitalRateExpr}. The key difference here is
that rather than using parentheses to wrap the the state variable,
square brackets are used. Below is an example of a P kernel that
combines survival and growth.

\texttt{P{[}size\_2,\ size\_1{]}\ =\ s{[}size\_1{]}\ *\ g{[}size\_2,\ size\_1{]}}

If the model contains hierarchical effects, then the suffix is again
appended to each term that the hierarchical effect modifies. Consider a
model with a random effect for year with the suffix \texttt{yr} from
\textbf{EnvironmentVariables}. The P kernel from above then becomes

\texttt{P\_yr{[}size\_2,\ size\_1{]}\ =\ s\_yr{[}size\_1{]}\ *\ g\_yr{[}size\_2,\ size\_1{]}}

Note that the the state variables are not modified with the suffix, only
the vital rates and kernels.

\hypertarget{examples}{\section{Worked Examples}\label{examples}}

\subsection{Non-hierarchical models}\label{non-hierarchical-models}

\subsection{Hierarchical models}\label{hierarchical-models}

\hypertarget{transforms-funs}{\section{Common Transformations and
Functions}\label{transforms-funs}}

\section{PDF Abbreviations}\label{pdf-abbrevs}


\end{document}
